% *-----------------------------------------------------------*
% *         Page de Garde (bas� sur le mod�le UPMC)
% * Derni�re modif le : 2012/02/13
% *-----------------------------------------------------------*
% * README: 
% *   - se compile avec pdflatex
% *   - l'ent�te de ce fichier sert pour g�n�rer une page
% *     unique. Ce texte latex peut s'inclure (partie comprise 
% *     entre les lignes et  ) dans un document existant.
% *-----------------------------------------------------------*

% - - - - - - - ent�te pour avoir une page unique 
\documentclass[11pt]{book}
\usepackage[french]{babel}
\usepackage[latin1]{inputenc}
\usepackage{graphicx}

\setlength{\textwidth}{16cm}
\setlength{\textheight}{25cm}
\setlength{\oddsidemargin}{-0cm}
\setlength{\topmargin}{-1cm}
\title{}     
\date{} 
\author{}
\begin{document}

% - - - - - - - d�but de la page 
\thispagestyle{empty}

\includegraphics[scale=0.12]{upmc-logo.png}

{\large

\vspace*{1cm}

\begin{center}

{\bf TH\`ESE DE DOCTORAT DE \\ l'UNIVERSIT\'E PIERRE ET MARIE CURIE}

\vspace*{0.5cm}

Sp\'ecialit\'e \\ [2ex]
{\bf Informatique}\ \\ 

\vspace*{0.5cm}

�cole doctorale Informatique, T�l�communications et �lectronique (Paris)

\vspace*{1cm}


Pr\'esent\'ee par \ \\


\vspace*{0.5cm}


{\Large {\bf Pr�nom NOM}}

\vspace*{1cm}
Pour obtenir le grade de \ \\[1ex]
{\bf DOCTEUR de l'UNIVERSIT\'E PIERRE ET MARIE CURIE} \ \\

\vspace*{1cm}

\end{center}

\flushleft{Sujet de la th\`ese :\ \\
\ \\
{\Large {\bf Le titre de ma Th�se \\ }}
  

\vspace*{1.5cm} 
\flushleft{soutenue le 30 f�vrier 2042}\\[2ex]
\flushleft{devant le jury compos� de :  }\\[1ex]
\flushleft{\begin{tabular}{r@{\ }ll}
  & M. Pr�nom {\sc Nom} & Directeur de th�se\\
  & M. Pr�nom {\sc Nom} & Rapporteur \\
  & M. Pr�nom {\sc Nom} & Rapporteur  \\
  & M. Pr�nom {\sc Nom} & Examinateur  \\
  & M. Pr�nom {\sc Nom} & Examinateur  \\
\end{tabular}}

}
% - - - - - - - fin de la page 
\end{document}
