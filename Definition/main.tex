\chapter{Flots de liens: extensions temporelle des graphes}
\minitoc
\label{chap:def_flot}

\section{Définition}
\label{sec:definition}
Un flot de liens est défini comme un triplet: $\mathcal{L}=(T,V,E)$, où $T=[\alpha, \omega]$ est un intervalle de temps, $V$ un ensemble de $n$ n\oe uds et $E\subseteq T\times T \times V \times V$ un ensemble de $m$ liens.
Les liens de $E$ sont des quadruplets $(b,e,u,v)$, signifiant que la paire de n\oe uds $(u, v)$ est connectée sur l'intervalle $[b,e] \subseteq [\alpha,\omega]$.
Nous considérons un flot non orienté, \emph{i.e.} $(b,e,u,v)=(b,e,v,u)$, et sans boucle, \emph{i.e.} $u \neq v$.


Nous dénotons la durée du flot par $\bar{L}=\omega-\alpha$.
$\beta(E)= min_{(b,e,u,v) \in E} (b)$ et $\psi(E)= max_{(b,e,u,v) \in E} (e)$ sont respectivement l'apparition du premier lien et la disparition du dernier lien.

Un flot de liens est simple si pour tout $(b,e,u,v) \in E$ et $(b',e',u, v) \in E$, $[b,e]\cap [b', e'] = \emptyset$.
La simplification d'un flot de liens $\sigma=L$.

$V(E')= {u}_{(b,e,u,v) \ in E'}$ sommets induits par un ensemble de liens.


$\xi(L,\Delta)$ ajout de $\Delta$ à chaque lien.

\inline{Que des flots avec durées ou bien delta densité?}
\section{Sous-flot induit}
Sous flot induit par un ensemble de lien $E'$: $L(E')=([\beta(E'),\psi(E'),V(E'),E'])$.


 
Sous flot induit par un ensemble de paire n\oe uds  $S \in V^ 2$: $L(S)$.
$L(S)=([\beta(E'),\psi(E')],V',E')$ avec $E'= \{(b,e,u,v) \in E, (u,v) \in S\}$.
Par convention, on note $L(v)= L(\{v\}\times V)$.

Sous flot induit par un intervalle de temps $T'=[\alpha', \omega'],\ T' \subseteq T$: $L_{\alpha'..\omega'}$.
$L_{\alpha'..\omega'}=([\alpha', \omega'],V(E'),E')$ avec $E'= \{(b',e',u,v),\ \exists (b,e,u,v) \in E,\ b'= max(b,\alpha'),\ e'=min(e,\omega')\}$.
Par convention, on note $L_{t..t}=L_{t}$


Il est aussi possible de combiner ces notions.
Par exemple avec $V' \subset V$, $L_{\alpha'..\omega'}(V'^2)$ est le sous flot correspondant au lien entre les n\oe uds de $V'$ sur l'intervalle $[\alpha', \omega']$.


\section{Degré et densité}
Beaucoup de notions sont développées autour de cet objet.
Degré temporelle d'un n\oe ud $u$:

\begin{equation}
d_t(u)= |L_t(v)| = |\{(b,e,u,v) \in E,\ b \leq t \leq e\}|
\end{equation}

Par extension pour un ensemble de sommets $V'$ :

\begin{equation}
d_t(V')= |L_{t}(V'^2)| = |\{(b,e,u,v) \in E,\ u,v \in V',\ b \leq t \leq e\}|
\end{equation}

\inline{Degré interne maintenant?}

\begin{equation}
d_t(E')= |\{(b,e,u,v) \in E',\ b \leq t \leq e\}|
\end{equation}

Il est possible d'intégrer sur l'ensemble du temps 
\begin{equation}
D_{\alpha..\omega}(v)= \int_{\alpha}^{\omega}d(v,t) dt=D(v)
\end{equation}

Par convention, on notera le degré moyen de $v$: $d(v)= \langle d(v,t) \rangle=\frac{D(v)}{\omega-\alpha}$.
Il en va de même pour les autres degrés.

C'est le cas de la densité:
\begin{equation}
\delta(L)= \frac{2\sum_{l \in E}}{n(n-1)(\bar{L})}
\end{equation}





\chapter*{Liste des notations pour les flots de liens}

\addstarredchapter{Liste des notations pour les flots de liens}

\begin{center}
\begin{tabular}{|c|c|}
\hline Symbole & description \\
\hline $L$ & Flot de liens \\ 
$T$ & intervalle de temps  \\
$V$ & ensemble de n\oe uds\\
$E$ & ensemble de liens: $(b,e,u,v)$ \\
$n$ & nombre de n\oe uds  \\
$|L|,|E|$ & nombre de liens dans le flot \\
$\beta(E)$ & temps d'apparition du premier lien\\
$\psi(E)$ & temps de disparition du dernier lien\\
$\xi(L,\Delta)$ & Flot de liens où chaque lien dure $\Delta$\\
$L(V'^2)$ & sous flot induits par les n\oe uds de $V'$ \\
$L_{t..t'}$ & sous flot induits par l'intervalle $[t,t']$ \\
$d_t(v)$ & degré de $v$ à l'instant $t$\\
$d(v)$ & degré moyen $v$ sur $T$\\
$\delta(L)$ & densité du flot\\
$\delta_{\Delta}(L)$ & densité du flot ou chaque lien dure $\Delta$\\

\hline
\end{tabular} 
\end{center}

