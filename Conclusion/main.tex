\chapter{Bilan}
\label{Conclusion}

\section{Résumé et contributions apportées}

Dans cette thèse, nous nous efforçons à poursuivre les travaux de descriptions et de capture de la structure des réseaux dynamiques tout en nous basant sur une base formelle la plus solide possible.
Nous nous sommes basé sur le formalisme de flot de liens qui fait le lien entre les propriétés temporelles et topologiques des réseaux.
Ce formalisme est suffisamment expressif pour définir énormément de notions.
Dans cette thèse nous nous sommes focalisés sur les notions de degrés et de densités, que nous avons présentés dans le chapitre~\ref{chap:def_flot}.
En plus d'aider à définir ces notions, nous les avons également implémentées dans une libraire de manipulation de flots de liens\,\footnote{\url{A FAIRE}}\note{A faire}.
Ainsi avec d'une part un formalisme clairement défini et d'autre part une implémentation accessible, on peut espérer faciliter l'utilisation de la densité mais surtout permettre à quiconque d'étendre ce formalisme en définissant d'autres métriques.

\`A partir de ce formalisme, il nous a été possible de commencer à décrire les structures existantes dans les flots de liens, dans le chapitre~\ref{chap:mailing}.
Pour ce faire, il est nécessaire de connaître un objet d'étude où la structure est connue.
Il n'existe quasiment\,\footnote{Un jeu de données représentant le forum reddit a été rendu publique juillet 2015.} aucun réseau réel évoluant dans le temps qui ait une structure connue au niveau des liens.
C'est pourquoi, nous avons collecté un jeu de données regroupant les courriels d'une liste de discussions.
Ce jeu de données est très intéressant par sa temporalité, presque 20 ans, et sa précision, de l'ordre de la seconde.
Il est donc possible que les caractéristiques du réseau aient changées au cours du temps.
Grâce à ces informations, nous avons pu étudier la structure de discussions du point de vue de la densité.
Nous avons notamment pu mettre en évidence que les discussions sont plus densément connectées en interne qu'avec le reste du réseau.
Ces travaux ont donné lieu à une publication dans une conférence internationale~\cite{Gaumont2016}.
Suite à ce constat, nous avons essayé retrouver automatiquement les discussions via une projection du flot de lien en un graphe statique.
Bien que la partition trouvée ne correspondent pas exactement aux discussions, il semble cependant que les groupes de la partition n'aient pas des caractéristiques triviales.


\bigskip
Dans le chapitre~\ref{groupesDense}, nous étudions plus en détails comment valider les groupes de liens détectés dans une projection du flot de liens en un graphe statique pour ne retenir que les groupes pertinents.
En effet, il se peut que parmi tout ces groupes de liens certains ne soient pas pertinents car ils sont obtenus via un graphe statique.

Comme nous l'avons constaté dans les chapitres~\ref{chap:def_flot} et \ref{chap:mailing}, la densité permet d'évaluer un groupe mais, seule, elle ne suffit pas définir un groupe pertinent.
En effet, un groupe n'est pertinent que par rapport au reste du flot de liens.
Par exemple, un groupe ayant une densité de $0.4$ est très pertinent dans un flot de liens ayant une densité globale de $0.01$ mais il ne l'est pas du tout si le flot de liens à une densité de $0.6$.
Pour évaluer un groupe, il est donc nécessaire de prendre en compte le contexte dans lequel il apparait.
Pour ce faire, nous avons définis trois voisinage constitués de sous-flots qui change du sous-flot du groupe sur le temps de début, la duré ou l'ensemble de n\oe uds.
Un groupe est ensuite jugé pertinent si le groupe est plus dense que ses voisinages.

Nous avons appliqué notre méthode sur quatre jeux de données et nous avons pu valider manuellement quelques groupes grâce aux méta-données présentes.
La structure des groupes pertinents est une partition partition partielle des liens du flot de liens.
Cependant en observant les n\oe uds induits et les instants de présences des groues, nous avons constaté que les groupes forment une structure très chevauchante sur la n\oe uds et uniquement partiellement chevauchante sur le temps.
Nous avons testé si une méthode statique pouvait retrouver cette même structure chevauchante et nous avons constaté qu'uniquement quelques groupes pertinents étaient capturé par la méthode statique.
Nous avons donc mis avant une structure nouvelle via la partition partielle que sont les groupes pertinents.
Ces travaux ont donné lieu à une première publication dans une conférence nationale~\cite{Gaumont2016} et sont soumis au journal \emph{Social Network Analysis and Mining}.


\bigskip

\`A partir de cette structure partielle, il est intéressant d'essayer de détecter une partition complète du flot de liens.
La détection de structure dans un réseau dynamique ou non nécessite un moyen d'évaluer le résultat obtenu.
C'est pourquoi nous avons étudié plus en détails la détection de communautés sous forme de partition de liens et ceux dans le cas statique et dynamique.
Comme nous l'avons vu dans le chapitre~\ref{chap:Expected_Node}, il existe assez peu de méthodes pour la détection de communauté sous la forme de partition de liens et aucune ne s'est vraiment attachée à reproduire ce qui a fait le succès de la modularité.
C'est pourquoi avons proposé une nouvelle méthode pour l'évaluation d'une partition de liens qui se base également sur la notion de modèle nul.
Un groupe de liens est ainsi une bonne communauté si il induit moins de n\oe uds qu'attendu dans le modèle nul.
De plus, un groupe de liens est également évalué en fonction de voisinage.
Cette fois ci, un voisinage est bon lorsqu'il consiste en moins de n\oe uds adjacents qu'attendu dans le modèle nul car le vosinage doit être peu dense.

Nous avons mené des tests sur des graphes ayant une structure simple, une clique ou deux cliques se chevauchant, et sur des graphes générés aléatoirement.
Sur ces exemples, notre fonction de qualité, \emph{Expected Nodes} est la seule à différencier la vérité de terrain d'autres partitions moins pertinentes.
Ces travaux ont été publié dans une conférence nationale~\cite{Gaumont2014} et une conférence internationale~\cite{Gaumont2015}.
Fort de ces premiers constats, nous avons implémenté un premier algorithme d'optimisation d'\emph{Expected Nodes}.
Cependant malgré nos efforts pour rendre plus efficace l'évaluation d'\emph{Expected Nodes}, notre algorithme n'est pas adapter à l'étude de graphes réels.

\bigskip

Fort de nos observations dans le cas dynamique et de notre proposition dans le cas statique, nous avons considéré les partitions de liens dans les flots de liens dans le chapitre~\ref{versQualite}.
Pour ce faire, nous nous sommes intéressé à la génération de flot de liens avec une structure donnée et à la manière d'évaluer une telle partition.
Pour générer un flot de liens, nous nous basons sur le \emph{Stochactic Block Model}.
Ainsi, les liens entre deux n\oe uds sont générés dans le cadre d'une communauté qu'ils partagent et le temps de début des liens dépendent de cette communauté commune.
Ainsi, il est possible de générer des flots de liens très variés tout en ayant une vérité de terrains sur les liens.

\`A partir de ce mécanisme de génération, nous avons d'une part mis en avant l'inefficacité de certaines méthodes statiques à retrouver la structure générée dans une projection du flot de liens.
D'autre parts, nous avons esquissé de premier travaux étudiant des extensions de la modularité dans les flots de liens.
Ces résultats, bien que non concluant, nous ont permis d'entre voir les nombreuses possibilités pour définir une fonction de qualité dans les flots de liens.

\section{Perspectives}

L'ensemble de ces travaux ouvre de nombreuses pistes de recherche que nous déjà avons évoqué de manière individuelle à la fin de chaque chapitre.
De manière plus générale, il y a deux pistes distinctes aux travaux de cette thèse: l'extension du formalisme de flot de liens et l'approfondissement des métriques et applications dans les flots de liens.

\subsection{Extensions du formalisme de flot de liens}

Le formalisme de flot de liens que nous avons utilisé est assez général pour représenter beaucoup de situations différentes mais il y en a cependant certaines qui ne sont pas parfaitement représentées par un flot de liens classique.
Il s'agit d'extensions similaires à celles existantes dans les graphes.
Il serait intéressant de considérer l'orientation des liens et les flots de liens biparties.
Ces ajouts peuvent se faire de manière assez simple: respectivement $(b,e,u,v)\neq (b,e,v,u)$ et $\forall (b,e,u,v) \in E, partie(u)\neq partie(v)$ avec $partie(u) \in \{1,2\}$.
Il est alors seulement nécessaire d'adapter les notions de degré et de densité pour tenir compte de ces spécificités.

L'ajout de poids sur les liens est bien sûr une extension possible mais il y a cependant deux extensions possibles: soit le poids d'un lien est fixe soit il peut changer au cours du temps.
Le second cas englobe le premier et permet de représenter beaucoup plus de situations comme par exemple les flots de liens non simples, \emph{i.e.} plusieurs liens en même temps entre deux n\oe uds.

Nous avons principalement discuté de la notion de degré et de densité dans cette thèse.
Il existe bien évidement de nombreuses autres notions qui ont déjà été définies comme les notions de chemins et de centralité.
Il serait d'intégrer ces notions dans le formalisme de flots de liens afin de rendre l'ensemble des notions cohérentes entre elle.
Par exemple, il serait intéressant de définir une mesure de proximité à partir du degré temporel.

En plus de l'extension des métriques de la théorie  des graphes, il est intéressant de considérer d'autres type de mesures.
En effet, les flots de liens regroupe de l'information structurelle et temporelle.
Or avec les mesures de la théorie des graphes, nous apportons principalement de l'information structurelle.
\`A l'opposé de ce processus, il existe la théorie du signal qui est focalisée sur l'information temporelle.
Il serait très intéressant de tirer parti de ce domaine pour enrichir les méthodes d'analyses des flots de liens.
Il est par exemple peut être possible de considérer un flot de liens comme un signal.
Il serait alors très intéressant d'étudier voir de combiner les signaux de différents sous-flots.

\subsection{Approfondissement des applications dans les flots de liens}
Dans cette thèse, nous avons étudié les groupes de liens comme structure communautaire.
Cette notion, bien que très spécifique, mène à de nombreuses perspectives sur la notion de communauté et la génération de flots de liens, comme nous l'avons évoqué dans à la fin du chapitre~\ref{groupesDense} et dans le chapitre~\ref{versQualite}.
Mais il y a bien d'autres pistes.
Il serait, par exemple, intéressant de comparer des groupes de liens.
En effet, deux groupes de liens distincts peuvent partager exactement les même n\oe uds et apparaître dans le même intervalle et donc être similaire.
Le simple indice de jaccard n'est donc pas suffisant pour comparer deux groupes de liens.
Il faut donc un indice tirant parti de la topologie et du temps.
Si une telle relation existe, alors il serait très intéressant d'étudier le graphe des relations entre les groupes pertinents ou les groupes d'une partition de liens.
Il s'agirait d'un graphe orienté par le temps et pondéré par la similarité entre les groupes.
On peut espérer trouver dans un tel graphe les groupes de liens reliant des n\oe uds similaires.
Il serait aussi surement possible de faire un parallèle avec les évolutions des communautés dans une série de graphe.
Par exemple, un n\oe uds ayant deux fils (resp. parents) représenterait alors la séparation d'une communauté en deux (resp. la fusion de deux communautés).

Par ailleurs dans le chapitre~\ref{versQualite}, nous avons tenté de définir une fonction de qualité globale pour évaluer une partition de lien.
Il serait intéressant de poursuivre la piste inverse et d'essayer de définir une fonction de qualité locale.
Le but serait alors d'avoir une fonction ne prenant en compte que le voisinage immédiat d'un groupe de liens pour l'évaluer contrairement à notre notion de pertinence qui prends en compte tout le voisinage en compte.

Enfin, il serait intéressant d'étudier les problèmes connexes à la détection de communauté de liens.
La compression de l'information en est un exemple car une structure communautaire peut être vu comme un \emph{résumé} de l'objet d'étude initial.
Avec ce point de vue, il pose alors la question de la perte de l'information induite par l'utilisation d'une structure communautaire pour représenter le flot de liens globales.
Un autre exemple et la prédiction de liens.
En effet dans un graphe, la prédiction des liens est améliorée si la structure communautaire est connue.
Dans les flots de liens, la situation est tout de fois plus complexe car, en plus de prédire une interaction, il faut également prédire son temps d'apparition.



\bigskip

\resume{
Au cours de cette thèse, nous nous avons étudié le problème de la détection de communauté dans un contexte dynamique.
La prise en compte de la dynamique est encore récente.
Il n'existe aucune taxonomie commune et chaque proposition nécessite de redéfinir ces propres notions.
Il s'agit surement d'un frein pour le développement de méthode de détection de communautés dans un contexte dynamique.
Très loin d'avoir résolu le problème de détection, les travaux de cette thèse permettent surtout de mieux définir le problème.
Il est ainsi important de comprendre les différences entre les formalismes: série de graphes, graphes temporels et flots de liens.
\`A partir de ce premier choix, il devient alors possible d'essayer de décrire une communauté dans ce contexte.
Nous faisons le parti pris de considérer le formalisme de flots de liens et les partitions de liens comme structures communautaires.
De ce présupposé, nous avons défini plusieurs métriques et décrit une structure communautaire existante avant de proposer une méthode de détection partielle.
Enfin nous avons esquissé de premier travaux de générations de flots de liens avec une structure.\\
\hspace*{1.4em} Afin de mener ces travaux, nous avons constamment essayer de partager les trois ingrédients majeurs à la recherche dans ce domaine: les données, une implémentation et la théorie.
Avec cette démarche, nous espérons ainsi faciliter les échanges entre chercheur pour faire avancer la recherche dans ce domaine où encore beaucoup reste à faire tant au niveau théorique que pratique.
}