\chapter{Conclusion et perspectives}




L'étude des réseaux complexes est un très vaste sujet d'étude et les graphes y ont une place particulière.
Les questions posées sur ces réseaux ont permis la formalisation de nouveaux problèmes et ont donné lieu à de nombreux axes de recherche dans la théorie des graphes.
De plus, des jeux de données de tailles très variées sont depuis plusieurs années disponibles et permettent de tester l'efficacité des algorithmes en terme de rapidité et de qualité de la solution trouvée.
La recherche de structures dans les graphes est un des sujets extensivement étudiés dans la littérature, comme nous l'avons vu dans la section~\ref{sec:intro_communaute}.
La grande majorité des méthodes considèrent un graphe statique, ce qui présuppose que le réseau étudié peut être représenté par un graphe.
C'est le cas par exemple du réseau de personne sur Facebook.
Il est possible de représenter par un graphe l'ensemble des liens d'amitiés qui existent aujourd'hui dans le réseau de Facebook; même si ce graphe est très important avec 1 milliard d'utilisateur.
Cette hypothèse n'est cependant pas vérifiée si le réseau évolue vite dans le temps, ce qui est le cas des réseaux de télécommunications ou même du réseau d'interaction des personnes sur Facebook.
Dans ce genre de réseaux, le réseau évolue en fonction du temps et il n'est plus possible de le représenter par un graphe.
C'est pourquoi il faut tenir compte de la temporalité du réseau.


Comme nous l'avons vu dans le chapitre~\ref{chap:etat_art}, il existe déjà plusieurs formalismes qui prennent en compte l'information temporelle.
En particulier, il y a deux formalismes capturant l'ensemble de l'information temporelles: les graphes temporels et les flots de liens.
Bien qu'équivalent, ces deux formalismes modélisent des situations différentes.
Dans les graphes temporels, il existe une structure de graphes à chaque instant et par conséquent les liens durent assez longtemps.
C'est le cas d'un réseau d'amitié lorsqu'il est étudié sur une longue période.
Dans ce genre de réseau, il est plus judicieux de détecter une première structure communautaire puis de la faire évoluer de manière continue en fonction des évolutions du réseau.
Le point de vue est donc plus centré sur les n\oe uds et leur évolution.
Un lien dure alors quelques mois ou années avant de disparaître.
Dans les flots de liens, la dynamique est beaucoup plus rapide et il n'existe \emph{a priori} aucune structure de graphe pertinente à un instant donné.
C'est le cas des réseaux d'interactions où il est très rare d'interagir avec quelqu'un pendant plus d'une heure.
Dans cette situation, il est plus aisé de considérer les liens comme objets d'études.
C'est ce type de situation que nous considérons et c'est pourquoi nous utilisons le formalisme de flot de liens.


La théorie des graphes a évolué et mûri pendant plusieurs dizaines d'années avant que l'émergence de grand graphe ne pose la question de la potentielle structures communautaires.
Toute la base formelle de la théorie des graphes était donc déjà développée lors de l'apparition des méthodes de détections de communautés dans les années 2000.
La formalisation des graphes a grandement aidé l'émergence des premiers travaux et des premières mesures décrivant les structures communautaires dans les réseaux réelles.
Le processus de recherche est complètement inverse dans le cas des réseaux dynamiques.
Ils existent de plus en plus de données et d'applications possibles alors qu'il n'existait jusque récemment aucune théorie générale sur laquelle s'appuyer.
C'est pourquoi les solutions apportées dépendent beaucoup des cas d'applications et sont difficilement applicables et compréhensibles dans d'autres contextes.
Cependant, quelques caractéristiques semblent être communes aux réseaux provenant de données réelles, \emph{e.g.} une distribution inter-contacts hétérogènes.
Ce résultat est à mettre en parallèle avec les premières similitudes observées dans les graphes de terrains.

\bigskip


Dans cette thèse, nous nous efforçons de poursuivre ces efforts de descriptions et de capture de la structure des réseaux dynamiques tout en nous basant sur une base formelle la plus solide possible.
Nous nous sommes donc basé sur le formalisme de flot de liens qui fait le lien entre les propriétés temporelles et topologiques des réseaux.
Ce formalisme est suffisamment expressif pour définir énormément de notions.
Dans cette thèse nous nous sommes focalisés sur les notions de degrés et de densités, que nous avons présentés dans le chapitre~\ref{chap:def_flot}.
En plus d'aider à définir ces notions, nous les avons également implémentées dans une libraire de manipulation de flots de liens\,\footnote{\url{A FAIRE}}\note{A faire}.
Ainsi avec d'une part un formalisme clairement défini et d'autre part une implémentation accessible, on peut espérer faciliter l'utilisation de la densité mais surtout permettre à quiconque d'étendre ce formalisme en définissant d'autres métriques.

\`A partir de ce formalisme, il nous a été possible de commencer à décrire les structures existantes dans les flots de liens, dans le chapitre~\ref{chap:mailing}.
Pour ce faire, il est nécessaire de connaître un objet d'étude où la structure est connue.
Il n'existe quasiment\,\footnote{Un jeu de données représentant le forum reddit a été rendu publique juillet 2015.} aucun réseau réel évoluant dans le temps qui ait une structure connue au niveau des liens.
C'est pourquoi, nous avons collecté un nouveau jeu de données regroupant les courriels d'une liste de discussions.
Ce jeu de données est très intéressant par sa temporalité et sa précision.
Comme il dure presque 20 ans, il est donc possible que les caractéristiques du réseau aient changées au cours de laps de temps.
Ce jeu de données est par ailleurs suffisamment précis car la minute d’envoi ainsi que la discussion de chaque courriel sont connues.
Grâce à ces informations, nous avons pu étudier la structure de discussions du point de vue de la densité.
Nous avons notamment pu mettre en évidence que les discussions sont plus densément connectées en interne qu'avec le reste du réseau.
Ces travaux ont donné lieu à une publication dans une conférence internationale~\cite{Gaumont2016}.
Suite à ce constat, nous avons essayé retrouver automatiquement les discussions.
Bien que la partition trouvées ne correspondent pas exactement aux discussions, il semble cependant que les groupes de la partition n'aient pas des caractéristiques triviales.
Ces travaux ont ainsi donné lieu à la partie la plus importante de cette thèse sur la détection de groupes denses.

\bigskip

Pour détecter une partition de liens dans le flots de liens, nous nous avons tout d'abord défini une projection du flot de liens en un graphe statique.
Dans cette projection, un lien du flot de liens est représenté par un n\oe uds dans la projection.
Un lien existe entre deux n\oe uds de la projection si les liens correspondant dans le flot de liens sont relié au même n\oe ud durant au moins un instant.
Sur cette projection, nous appliquons un algorithme de détection de communautés et obtenons une partition de liens.
Il se peut que parmi tout ces groupes de liens certains ne soient pas pertinents car ils sont obtenus via une optimisation de la modularité sur un graphe agrégé.
Il est donc nécessaire de pouvoir distinguer les groupes pertinents des autres groupes.


Comme nous l'avons constaté dans les chapitres~\ref{chap:def_flot} et \ref{chap:mailing}, la densité permet d'évaluer un groupe mais, seule, elle ne suffit pas définir un groupe pertinent.
En effet, un groupe n'est pertinent que par rapport au reste du flot de liens.
Par exemple, un groupe ayant une densité de $0.4$ est très pertinent dans un flot de liens ayant une densité globale de $0.01$ mais il ne l'est pas du tout si le flot de liens à une densité de $0.6$.
Pour évaluer un groupe, il est donc nécessaire de prendre en compte le contexte dans lequel il apparait.
Pour ce faire, nous avons définis trois voisinage constitués de sous-flots qui change du sous-flot du groupe sur le temps de début, la duré ou l'ensemble de n\oe uds.
Un groupe est ensuite jugé pertinent si le groupe est plus dense que ses voisinages.

Nous avons appliqué notre méthode sur quatre jeux de données et nous avons pu valider manuellement quelques groupes grâce aux méta-données présentes.
La structure des groupes pertinents est une partition partition partielle des liens du flot de liens.
Cependant en observant les n\oe uds induits et les instants de présences des groues, nous avons constaté que les groupes forment une structure très chevauchante sur la n\oe uds et uniquement partiellement chevauchante sur le temps.
Nous avons testé si une méthode statique pouvait retrouver cette même structure chevauchante et nous avons constaté qu'uniquement quelques groupes pertinents étaient capturé par la méthode statique.
Nous avons donc mis avant une structure nouvelle via la partition partielle que sont les groupes pertinents.
Ces travaux ont donné lieu à une première publication dans une conférence nationale~\cite{Gaumont2016} et sont soumis au journal \emph{Social Network Analysis and Mining}.


\bigskip

\`A partir de cette structure partielle, il est intéressant d'essayer de détecter une partition complète du flot de liens.
La détection de structure dans un réseau dynamique ou non nécessite un moyen d'évaluer le résultat obtenu.
C'est pourquoi nous avons étudié plus en détails la détection de communautés sous forme de partition de liens et ceux dans le cas statique et dynamique.
Comme nous l'avons vu dans le chapitre~\ref{chap:Expected_Node}, il existe assez peu de méthodes pour la détection de communauté sous la forme de partition de liens et aucune ne s'est vraiment attachée à reproduire ce qui a fait le succès de la modularité.
C'est pourquoi avons proposé une nouvelle méthode pour l'évaluation d'une partition de liens qui se base également sur la notion de modèle nul.
Un groupe de liens est ainsi une bonne communauté si il induit moins de n\oe uds qu'attendu dans le modèle nul.
De plus, un groupe de liens est également évalué en fonction de voisinage.
Cette fois ci, un voisinage est bon lorsqu'il consiste en moins de n\oe uds adjacents qu'attendu dans le modèle nul car le vosinage doit être peu dense.

Nous avons mené des tests sur des graphes ayant une structure simple, une clique ou deux cliques se chevauchant, et sur des graphes générés aléatoirement.
Sur ces exemples, notre fonction de qualité, \emph{Expected Nodes} est la seule à différencier la vérité de terrain d'autres partitions moins pertinentes.
Ces travaux ont été publié dans une conférence nationale~\cite{Gaumont2014} et une conférence internationale~\cite{Gaumont2015}.
Fort de ces premiers constats, nous avons implémenté un premier algorithme d'optimisation d'\emph{Expected Nodes}.
Cependant malgré nos efforts pour rendre plus efficace l'évaluation d'\emph{Expected Nodes}, notre algorithme est trop lent pour traiter des graphes réels.

\bigskip




\section{Perspectives}

L'ensemble de ces travaux ouvre de nombreuses pistes de recherche que nous déjà avons évoqué de manière individuelle à la fin de chaque chapitre.

Plein de mesures pour les flots de liens.
Chemin-> centralité

Prise en compte de l'orientation/poids/bipartie.


Fonction de qualité locale pour flots de liens

